\documentclass[11pt]{article}[times]

\topmargin 0.0cm 
\oddsidemargin 0.0in 
\evensidemargin0.0in
\textheight 22cm 
\textwidth  17cm 
\headheight 0in 
\headsep 0in
\parindent0in


\usepackage{amsmath}
\usepackage{amsfonts}
\usepackage{amssymb}
\usepackage{graphicx}
\graphicspath{{./Figures/}}
\usepackage{tikz}
\usepackage{amssymb}
\usepackage{circuitikz}
\usepackage{setspace}
\usepackage{pdfpages}
\begin{document}


\section*{Experiment 1: C-V Measurements}
\begin{figure}[htbp]
  \includegraphics[width=0.8\textwidth]{Figure_1.png} 
  \caption{Circuit Diagram for CV Measurements} 
  \label{fig:fig1} 
\end{figure}
The circuits shown in Figure \ref{fig:fig1}
is used to conduct the CV Measurements.

The open-loop frequency of the oscillator 
$f_0 = 1629.9 \pm 0.05 \text{kHz}$.

The "10 pF" capacitor $C_{10} = 10.48 \pm 0.005 pF$.

With $C_{10}$ connected, the frequency $f_{10} = 1467.4 \pm 0.05 \text{kHz}$.

From theories of oscillators, 
the circuit capacitance 
$C_0 = \frac{C_{10}}{\big(\frac{f_0^2}{f_{10}^2}\big)-1}$ = 44.
\section*{Experiment 2: I-V Measurements}

\subsection*{Reverse Bias}

\subsection*{Weak Forward Bias} 

\subsection*{Strong Forward Bias}



\appendix

\section*{Data Tables}












\end{document}